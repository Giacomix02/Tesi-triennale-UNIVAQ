\chapter{Conclusioni}
\pagestyle{plain}

\section{La mia esperienza con Unity e MRTK}
La mia esperienza con l'MRTK è stata in generale per lo più negativa, dettata da una scarsa documentazione ufficiale e anche poco materiale online di progetti di altri sviluppatori. I maggiori problemi si sono presentati quando ho dovuto costruire l'interfaccia grafica, in particolare nel far funzionare la scroll list e capire il funzionamento delle tab. Un'ulteriore difficoltà è stata reperire i prefab e gli esempi menzionati nella documentazione: spesso venivano mostrati come disponibili, ma risultavano difficilmente trovabili tramite fonti ufficiali. Anche il processo di compilazione per il visore non è dei migliori, con tempi di attesa per la prima build che si avvicinano alla mezz'ora, errori a fine compilazione abbastanza strani che si risolvono riavviando il pc, il visore o ricominciando da capo la compilazione, che come se non bastasse occupa uno spazio non indifferente sul disco e mi è capitato alcune volte di non aver spazio sufficente o addirittura di riempire completamente il disco. Altro punto a sfavore è la quantità di spazio richiesto per sviluppare l'applicazione, dato che bisogna installare componenti da Visual Studio Installer che pesano diversi GB. Altro problema che ho riscontrato, e che mi ha costetto a migrare l'intero progetto in un nuovo progetto Unity, è stato il fatto che installando dei componenti dalla Microsoft Mixed Reality Feature Tool, il progetto Unity non riusciva più a emulare dando errori di componenti mancanti, inoltre Unity, non permettendo di copiare e incollare la gerarchia senza creare dei prefab, ha reso più difficile e laborioso il processo di migrazione. Del resto la programmazione C\# in Unity è abbastanza lineare e ben documentata con un'ottima community online, che permette di risolvere la maggior parte dei problemi che si possono incontrare. Lato Hardware, gli Hololens sono un dispositivo molto interessante e con un potenziale enorme, anche se dotati di specifiche tecniche molto limitate nel 2025, punto a favore è che lo sviluppatore può controllarli in remoto accedendoci tramite IP, vedendo tutti i parametri vitali e anche una preview della telecamera, che permette di vedere cosa vede l'utente in tempo reale. Punto a sfavore invece è il FOV ("Field Of View") molto limitato e anche uno schermo che se non impostato correttamente potrebbe non avere i giusti colori o risultare sfocato.
\section{Sviluppi futuri dell'applicazione}
Si potrebbe migrare la parte di generazione di ingredienti e ricette su un server, così da poter creare un'applicazione multipiattaforma per qualsiasi dispositivo che supporti la realtà mista, dotando di un login l'utente in modo tale da poter andare a consultare le ricette salvate e gli ingredienti preferiti anche da altre piattaforme come smartphone o pc. Nel progetto attuale per gli Hololens si dovrebbe implementare la funzionalità di tracciare tramite le coordinate fornite da Gemini gli ingredienti nello spazio 3d, magari aggiungendo informazioni utili sui singoli ingredienti.

\section{Considerazioni finali}
L'esperienza di sviluppo con Unity e MRTK è stata complessa e ha richiesto un notevole impegno per superare le difficoltà legate alla documentazione e all'utilizzo dell'MRTK. Con lo sviluppo di questa applicazione ho capito il perchè non ci sono molti sviluppatori che si cimentano nello sviluppo di applicazioni per gli Hololens. Personalmente considero l'MRTK un progetto morto della Microsoft, che ha molti difetti e pochi pregi, surclassato da altri framework con una documentazione e processo di sviluppo migliore, come ad esempio quello di Meta utilizzato per lo sviluppo di applicazioni per il visori Meta Quest.