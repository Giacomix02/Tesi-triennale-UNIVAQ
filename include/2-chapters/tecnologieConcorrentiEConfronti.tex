\chapter{Tecnologie concorrenti, simili e confronti}
\pagestyle{plain}

\section{Hardware}
Andiamo a esplorare altri Hardware concorrenti o simili agli Hololens 2, facendo un confronto delle caratteristiche principali e del perchè sono stati scartati per essere utilizzati in questo progetto.
\subsection{Meta Quest 3}
Il Meta quest 3 è il visore di punta sul mercato consumer di Meta, è un visore standalone, ma che si può collegare anche al PC, con un sistema operativo basato su Android chiamato Meta Horizon OS. A differenza degli Hololens 2 ha un processore molto più potente, uno Snapdragon XR2 Gen2, con al massimo 512GB di memoria interna e 8Gb di Ram. A differenza degli Hololens2 ha uno schermo LCD con una risoluzione di 2064x2208 per occhio, e un refresh rate massimo di 120Hz. Dato che ha un display la realtà mista viene garantita tramite un sistema di passthrough, che permette di vedere il mondo reale attraverso le telecamere del visore, che riescono a ricostruire in maniera quasi perfetta la normale visione di un occhio umano. Dal punto di vista dello sviluppo di app questo avviene sempre con Unity e un pacchetto di sviluppo, ma a differenza degli Hololens 2 non è possibile accedere in nessun modo alle fotocamere frontali, diminuendo di molto le capacità delle applicazioni in realtà mista.
\subsection{HoloLens 1}
Gli HoloLens di prima generazione, usciti nel 2016, sono i predecessori degli Hololens 2, sono dotati di un Intel Atom x5-Z8100, 2Gb di RAM e 64Gb di memoria interna, le prestazioni, comparate con quelle del visore di nuova generazione, sono nettamente inferiori, inoltre questo visore è in fase di dismissione, in favore della nuova generazione. Quindi la scelta degli Hololens 2 è abbastanza banale, anche solo per il confronto di prestazioni.

\section{Tecnologie di Visualizzazione}
Oltre alla MR (Mixed Reality) esistono altre tecnologie di visualizzazione, le più importanti sono la VR (Virtual Reality) e la AR (Augmented Reality).
\subsection{Realtà Virtuale (VR) }
La Realtà Virtuale crea un ambiente virtuale completamente immersivo dove l'utente può muoversi e interagire, ad esempio prendendo oggetti. Ogni utente può avere anche un avatar e interagire con altri utenti nel mondo virtuale. Di solito questo tipo di tecnologia viene utilizzata con dei visori dotati di schermi che coprono completamente il campo visivo dell'utente, e che sono dotati di sensori di movimento, per permettere all'utente di muoversi in questo mondo virtuale. La VR viene utilizzata principalmente per videogiochi, ma anche per applicazioni professionali, come ad esempio la simulazione di ambienti per la formazione, ma anche per la creazione di riunioni aziendali virtuali.

\subsection{Realtà Aumentata (AR) }
Il più famoso esempio di Realtà Aumentata potrebbe essere il gioco Pokemon Go, dove gli utenti possono vedere i Pokemon nel mondo reale attraverso lo schermo del proprio smartphone per poi catturarli. La AR permette di sovrapporre oggetti virtuali al mondo reale, visualizzando informazioni extra come testi, indicazioni stradali, e oggetti 3d, ma non permette di interagire con essi.

\subsection{Differenze e confronto tra AR e MR}
La Realtà Aumentata e la Realtà Mista sono due tecnologie simili, ma con differenze fondamentali. La Realtà Aumentata sovrappone oggetti virtuali al mondo reale, ma non permette di interagire con essi in modo significativo. La Realtà Mista, invece, integra gli oggetti virtuali nel mondo reale in modo più profondo, permettendo all'utente di interagire con essi come se fossero parte del mondo reale. Ad esempio, in un'applicazione di Realtà Mista, un utente potrebbe afferrare un oggetto virtuale e spostarlo virtualmente nello spazio reale, mentre in un'applicazione di Realtà Aumentata l'oggetto virtuale rimarrebbe statico e non interagirebbe con l'ambiente circostante.

\subsection{Differenze e confronto tra VR e MR}
La Realtà Virtuale e la Realtà Mista sono due tecnologie che offrono esperienze immersive, ma con differenze significative. La Realtà Virtuale crea un ambiente completamente virtuale, isolando l'utente dal mondo reale, mentre la Realtà Mista integra elementi virtuali nel mondo reale, consentendo interazioni più naturali e intuitive per l'utente.

\cite{DifferenceMeta-AR-VR-MR} \cite{DifferenceIBM-AR-VR-MR}